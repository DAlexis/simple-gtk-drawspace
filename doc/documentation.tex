\documentclass[a4paper,11pt]{article}
\usepackage[T2A]{fontenc}
\usepackage[english,russian]{babel}
\usepackage[utf8]{inputenc} 
\usepackage[vmargin={3cm,3cm},hmargin={2cm,2cm}]{geometry}
\usepackage{graphicx}

\usepackage{amssymb}
\usepackage{amstext}
\usepackage{amsmath}
\usepackage[warn]{mathtext}
\usepackage{textcomp}

\usepackage{indentfirst}
\usepackage{wrapfig}

\usepackage{verbatim}

\title{Описание библиотеки \textbf{simple-gtk-drawspace}}

\begin{document}
\maketitle
\section{Введение}
Создание окна, используя GTK, Qt или другого тулкита для вывода графики "--- непростая задача для только начинающего программировать на C/C++. Библиотека \verb!simple-gtk-drawspace! предназначена для значительного упрощения этой задачи. Она представляет интерфейс, чем-то схожий с Borland BGI, позволяя рисовать графическими примитивами на создаваемом автоматически окне GTK. 
\section{Работа с библиотекой}
Взаимодействие с библиотекой происхдит через класс \verb!SimpleGTKDrawspace!, описанный в заголовочном файле \verb!simple-gtk-drawspace.h!
\subsection{Инициализация}
Для инициализации окна сначала необходимо создать экземпляр класса \verb!SimpleGTKDrawspace!. В качестве аргументов конструктору следует передать адреса аргументов функции \verb!main(int argc,! \verb!char* argv[])!. Это нужно, чтобы была возмоность передавать через аргументы командной строки дополнительные параметры библиотеке GTK+. <<Свои>> параметры после вызова конструктора будут автоматически исключены из \verb!argc! и  \verb!argv!. Также, можно применять конструктор без параметров, если передача дополнительных опций не потребуется в принципе.\par
Далее, необходимо вызвать функцию \verb!init()! у экземпляра класса \verb!SimpleGTKDrawspace!:

\begin{verbatim}
SimpleGTKDrawspace::init(unsigned int in_sizeX, unsigned int in_sizeY,
                           DrawFunction in_drawFunction)
\end{verbatim}
Где первые два аргумента "--- ширина и высота рабочей области, третий~"--- функция, которая будет осущетсвлять рисование. Она будет запущена немедленно в отдельном потоке. Её прототип определяется так:
\begin{verbatim}
typedef void (*DrawFunction) (SimpleGTKDrawspace* );
\end{verbatim}
Таким образом, для использования библиотеки \verb!simple-gtk-drawspace! программа должна иметь вид:
\begin{verbatim}
#include "simple-gtk-drawspace.h"
...
void draw(SimpleGTKDrawspace* window)
{
    ...
    window->... // Any drawing function
    ...
}
...

int main(int argc, char* argv[])
{
    ...
    SimpleGTKDrawspace window(&argc, &argv);
    window.init(500, 500, draw);
    ...
}
\end{verbatim}
\subsection{Графические примитивы}
Для рисования графики класс \verb!SimpleGTKDrawspace! имеет следующие функции:
\begin{itemize}
	\item \verb!void line(double x0, double y0, double x1, double y1)! "--- прямая линия из точки (x0,~y0) до точки (x1,~y1)
	\item \verb!void squareBrush(double x, double y, double size)! "--- квадрат с центром в точке (x,~y) со стороной size
	\item \verb!void squareBrushFilled(double x, double y, double size)! "--- закрашенный квадрат с центром в точке (x,~y) со стороной size
	\item \verb!void moveTo(double x, double y)! "--- переместить перо в точку с координатами (x,~y)
	\item \verb!void lineTo(double x, double y)! "--- провести линию из текущего положения пера до точки с координатами (x,~y). При этом (x,~y) становится новым положением пера.
	\item ...
\end{itemize}
\end{document}
