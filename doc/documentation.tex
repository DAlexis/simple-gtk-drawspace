\documentclass[a4paper,11pt]{article}
\usepackage[T2A]{fontenc}
\usepackage[english,russian]{babel}
\usepackage[utf8]{inputenc} 
\usepackage[vmargin={3cm,3cm},hmargin={2cm,2cm}]{geometry}
\usepackage{graphicx}

\usepackage{amssymb}
\usepackage{amstext}
\usepackage{amsmath}
\usepackage[warn]{mathtext}
\usepackage{textcomp}

\usepackage{indentfirst}
\usepackage{wrapfig}

\usepackage{verbatim}

\title{Описание библиотеки \textbf{simple-gtk-drawspace}}
\begin{document}
\maketitle
\section{Введение}
Создание окна, используя GTK, Qt или другого тулкита для вывода графики "--- непростая задача для только начинающего программировать на C/C++. Библиотека \verb!simple-gtk-drawspace! предназначена для значительного упрощения этой задачи. Она представляет интерфейс, схожий с Borland BGI, позволяя рисовать графическими примитивами на создаваемом автоматически окне GTK. 
\section{}
\end{document}
